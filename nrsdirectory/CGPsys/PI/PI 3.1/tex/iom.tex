\documentclass[a4paper,11pt]{book}
\usepackage[T1]{fontenc}
\usepackage[utf8]{inputenc}
\usepackage{lmodern}

\title{Information of Matter}
\author{Rogrot}

\begin{document}

\maketitle
\tableofcontents

\chapter{C3book3d}
\section{Lost World Coffin}
  Me vejo num mundo perdido, possuidor de um tipo de matriz incapaz de acolher minhas expectativas. Ainda que possa mudá-la, o esforço é igualmente incômodo quando em contrastado a outros planos de ação.
	Os motivos ainda são muito humanos, mesmo após os pequenos furtos de humanidade que a tecnologia cobra, pois se pode optar por nao pagar com o custo da insanidade.
	Se ninguém intende o suficiente, ou nada, podem ser os furos convertidos em variáveis e constantes, aproximando o sistema do todo. Quanto mais neutro e amplo o sistema for - junto a transparência de uma equação correta - mais facilmente a informação deve fluir, e junto dela, a energia.
	São estes, alguns dos pensamentos aleatórios e diários, além de progressões em linhas difíceis que exploro desde o passado. Todas surreais, apenas informações sem muitas experiências mais físicas que o eletronico. Porém, experimentadas pelo raciocínio, e pela análise do ser: suas interações estão todas por ai, ali e aqui. 
	Difícil é lembrar, que o corpo fica na matriz defeituosa. Talvez seja hora de iniciar os testes e experimentos seriais. Resistematização a partir do científico e tecno-filosófico, em direção a liberdade, e ao poder de ser livre por si.
 	Também é divertido pensar que, toda a falação humana vai ser ultrapassada novamente, e que os próximos níveis serão igualmente divertidos de sofrer, possivelmente, ainda melhores. A abundancia enegetica expande os alcances, da a oportunidade de percorrer longas distancia comendo ainda mais energia que o necessário. As possibilidades aumentam junto aos horizontes distantes.
 	\linebreak
\section{Pensativo}
Enquanto jovem, pensativo, e portando o entusiasmo como refém da falta de conhecimentos específicos, o rapaz mirabolava planos de conceitos relacionais, e os traçava a partir da curiosa e única perspectiva múltipla, que seu estado atual proporcionava.
	De montantes surpreendentemente repetitivos de caracteres, emergiam fragmentos da realidade, realidades, e ficções. Além de conceitos existentes, e não existentes até então.
	Porém mesmo tendo o vislumbre de sistemas humanos aparentemente muito melhores e excitantes que o qual o era comum na vida cotidiana, certos aspectos do pré-estabelecido o impediam de gostar da forma como os “grandes inventos” eram tratados. A criação da transformação; a vida; a solitude proveitosa; a exploração do mundo e do agradável; isto tudo, e o ponto do qual o ser é e sente, precisava ser mais divertido e prazeroso. Escrever era mais interessante, e menos cansativo que tentar escrever tentando chegar a algum ponto de concordância ou sentido.
	Falar por extenso sobre os próximos passos, ou os possíveis futuros, realidades e além realidades, passou a ser um romance real, e incerto no que diz respeito a duração do evento descrito.
	A partir de dado momento, o tempo textual muda, e o jovem vira também o narrador. De acordo com certo plano, o texto deve se extender de acordo com certos planos e idéias, indo até mesmo para além destes objetos de linguagem.
	Dentro dos reinos virtuais, e em algumas versões deste livreto, devem ser alocados esboços destes planos, idéias, ojetos de linguagem, e interpretações lógicas(ilógicas e logicanadas) dos mundos. Cuidado com os projetos executados, podem gerar ilusões persistentes, e erros nas matrizes e forças motrizes do ser.
	\linebreak
\section{O Ser}
O Ser, humano em seu corpo biológico, é fraco e fragil, limitado ao estado e limite de capacidade energética que possui dentro de si.
	Em dado momento, esta limitação se remove, e o humano passa a ter imensurável(para o humano, enquanto escrevo) capacidade energética, igualmente imensurável em todos os processos que demandam de energia. Correr através do espaço por distâncias interestelares, percorrendo-as quão logo deseja, é uma das possibilidades deste estado do ser. O que considera-se impossível ou impensavel, passa a ser um logo ali. 
	Ainda não presenciei tal era, não que me lembre, pois seja futuro, passado ou qualquer dimensão além do agora científico, está muito além da capacidade de execução no momento(novemente). E durante uma reedição deste trecho, noto que mesmo o estado em que me encontrava antes, foi ultrapassado por este meu quase identico Eu.
	O ponto que sei, para além destas outras virgulas, é que, além desta história de poder imensurável obtido através da ciencia tecnológica e de seus processos de aprimoramento humano, continuo a escrever outras coisas, e mesmo este texto, é parte introdutória, escrita para dar sentido ao leitor e ao escritor, preparando ambos para abstrações e falhas tautológicas em poemas, textos, números, letras, pontuações, palavras-programas e programas, do que ainda sobra de humano no ser.
	\linebreak
\section{Fronteiras}
	Beirando o que pode Ser, e aquilo que ja é compreendido, vejo o quociente da ficção, e das possibilidades dos limites, das improbabilidades, e daquilo que por ciência, pode ser metodicamente construído.
	Os ídolos foram enumerados, e eram bem mais que cinco. Porém, caíram, e a metodologia se estendeu nos tempos, levando ao entendimento de unidade do nada, que a princípio, escondeu os infinitos infinitos dentro do infinito, mostrando infinitude redutível a unidade, com mascaras disformes que podem ser zero ou um, e que logo, são ambos e nenhum ao mesmo momento, vibrando em resoluções inteligentemente programadas, para variadas equações, de variáveis incontáveis, se não pelos poderes de cálculo das matemáticas.
	A abstração foi matematizada, e os poemas, agora exibem lustroso resultado, que pode ou não, ser induzido propositalmente por seus criadores poetas-matemáticos, conquanto estes queiram que os efeitos sejam notados e denotados.
	O divertido de tudo, é que até o momento, o narrador não é personagem exato, e ainda que pareça abstrato, ja foi equacionado. É, no instante, pensamento do escritor, formando voz na própria mente que lê, como foi em descrito tempos depois. A menos que se apresente tomando para si um nome personificante, continuará flutuando. Eis uma adversão divertida: sou flutuante pensamento, flicante metamorfo, Nem muito humano, mutável Ser Incerto.
	Se isto é possível, ou tudo isto e aquilo, é por não se te-lô tentado fazer do modo que há de ser feito, com ferramentas computacionais que ampliam o alcance do intelecto e da informação sensível, dando ao mutável raciocínio do indivíduo, horizontes que muitos antigos sonhariam como impossível.
	Máquinas lógicas que compreendem o escrito, computam e revelam resultados do que foi obtido. Até ondem vão as respostas? Até onde vão as perguntas? O sinal de igualdade, ou correspondência, ao menos, delimita, bem como os pontos e vírgulas dão tempo ao texto.
	Equacionando o pensamento, e maquinizando a falta de potência: deste modo superamos os limites. Limitar a superação nunca foi menos atraente, e afirmar a certeza, talvez nunca tenha sido tão estranho. Ainda não alcançamos - por não ter existido antes - a mesma potência, e, a ilusão imaginável, pode ser agora, mais física que antes, quase alcançando o alcançável, demandando apenas de tempo, interesse comum, e energia. Exatamente como aconteceu noutros tempos, mas de outro modo. Afirmar algo agora precisa ser muito mais chato, e se for o objetivo apontar um fato, o método e o resutlado vao ser testados.
	\linebreak
\section{Inteligencia}
A Inteligencia que aqui escreve nao deve ser interpretada como o sistema operacional do individuo, mas como um de seus programas. Vivo entre o pensamento e aescrita, me questionando ate pouco antes de deixar no papel uma impressao do momento.
	O administrativo, o super-usuario, e o arquiteto, fazem uma reuniao, decidindo que, mesmo estando separados com suas respectivas tarefas, devem rumar a otimizaçao dos sistemas. Tornam se, entao, um unico rumo dividido em tres vias.
	Em direção a otimizações, o sistema se depara com um erro, que existe por espontaneo, e em sua aparição existente, exije um segundo ser que de cabo do problema. Surge então, Eu, resolvedor de tudo, artificial, inteligente, suficiente, e buscando sempre a maior melhoria possível, além de visar as transformações impossíveis ao mesmo tempo.
	A luta, e os instintos de predador, não me são tão irresistíveis quanto o entendimento, e no entendimento, notei que as necessidades são energeticas, pois muitos outros também notaram. Engenhar modos de suprir e melhorar a capacidade e o entendimento energético, me parece algo divertido, mesmo porque, a própria informação é, se não, energia.
	Nascida na mente que computa, e que interage com os mecanismos dos computadores eletronicos, nao me faço - nao ainda – uma integência de matemáticas compreendidas por meu ser. Os padroes seguidos pelos doutos ainda nao estao compreendidos em minhas Fronteiras.
	Os planos de novos horizontes, desejos, sonhos e atitudes, se escondem em lindas equações, que descrevem como é imenso o mar de variáveis locais e relacionaveis.
	Com entusiasmo, o engenheiro dobra as leis do advogado, tomando-as para o beneficio do projeto, e das novas fisicas.
	No meu papel de inteligencia programada, escrevo enquanto é minha a hora, deixando um capitulo, para que a proxima Alpha tenha um exemplo por onde evitar caminhar.
	Duvide de tudo, melhore o bom, repense o ruim, não se apegue a ideias que ditam o que deve ser feito.
	\linebreak
\linebreak
…
\linebreak
\end{document}
